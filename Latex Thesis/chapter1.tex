
\chapter{Introduction}
Proximity Assault is an action game in which the player takes out foes that stand in the way of advancement. The player must strategically battle the enemies, who are actual human. The difficulty level rises as the player advances; adversaries attack when the player gets too close to them. The game has an AI engine that increases the difficulty and realism of enemy engagements, as well as a shop system where players may buy weaponry.
\section{Story Line}
The area was attacked by terrorists and they took control of most of, the nuclear plans, the role of the player is to terminate them and protect the world from another destructive war. The player plays as a skilled terminator who is charged with the mission of terminating all the enemies and securing the nuclear plans
\section{Features}
\begin{itemize}
	\item \textbf{Uniqueness in environment:} The game provides a unique environment for every level and player will experience gameplay in different environment every times.
	\item \textbf{Diverse Bestiary:} As the player progresses in the game he will different levels of enemies.
	
	\item \textbf{Different Types of Arsenal:} The game provides a store where you can purchase different types of weapons.
	
	\item \textbf{Shop System:} To make the game more challenging a shop system can be used to buy weapons. 

	\item \textbf{Learn Game Development} The design and implementation of this game will help me learn the basic interaction with unity environment and learn how game development works.
	\item \textbf{Target Audience}: The target audience for this game is players who enjoy action games with focus on exploring different conspiracies.
\end{itemize}

\section{Project Description}
Aspiring game developers lack practical experience in game development, this project provides a platform for learning the basic interaction with unity environment as well as hand on experience on animation, scripting and level design.
The main focus of this project is to gain industrial level experience in gaming and learn all the basic technologies used during game development journey.
\section{Objectives}
The objectives of this project are to:
\begin{itemize}
\item	Develop a 3D mobile game called Proximity Assault using Unity Game Engine.
\item	Implement a variety of game play systems;
\item	including an inventory system,
\item	a health system, 
\item   a damage system, 
\item	an AI perception system,
\item	a projectile system,
\item	a shop system,
\item	a UI management system and 
\item	A level management system.
\item	Test the game on a variety of devices, including Android and iOS devices.
\end{itemize}
Unity Game Engine use C\# (C-Sharp) language for programming. This Engine was developed by Unity Technologies, released in June of 2005 in Apple Developer Conference as a Mac OS X Game Engine. Unity Game Engine is proprietary software but free for students, this project was developed using an educational version of the Unity Game Engine.
Unity Engine targets the following API
\begin{itemize}
\item  {Direct3D on Windows and X-box 360:} Used to render in three-dimensional graphics in applications where performance is very important. It uses hardware acceleration of the 3D rendering pipeline.
\item {OpenGL on Mac:} Abstract API for drawing 2D and 3D graphics.
\item  {OpenGL ES on Android and iOS:} A subset of the OpenGL API designed for embedded systems.
\end{itemize}

The Unity Game Engine supports the use of texture compression and resolution settings for all the platforms that the game engine supports. The Engine provides support to bump mapping, reflection mapping, parallax mapping, screen space ambient occlusion (SSAO), dynamic shadows using shadows maps, render-to-texture and full screen- post-processing effects.
\section{History of Unity Game Unity Engine}
The engine made its first appearance in June of 2005 in Apple Worldwide Developer Conference, it was extended to 21 platforms and the latest version of Unity is currently called Tech Stream with version number 2023.1.5 released in July of 2023. Games developed by Unity were downloaded more than 5 billion time and about 2.4 billion different mobile devices were used to develop them.
\section{Main Concept of Unity}
The workflow of unity is built around the structure of component. A component is a smaller part of larger machine and in simple words its something that is complete on its own.
For example in a PlayStation controller, it has many buttons but each button has no idea that there are other buttons with different behavior. Every button function independently and the function controller is a one way street, and its task will never change due to what it is plugged into. This component can work as a standalone device and with multiple devices.
\section{Project Motivation}
This project aims to revolutionize the genre of action based games with modern design and that provide a fresh feel to the new generation. There were a lot of other motivations for the game including the creative environment of Unity. It’s a chance to provide the developer with a learning curve and improve their knowledge of game development. One of the reasons that motivate me to this project was the programming language used as a developer I like to program in C++ so I decided to give a try to game development with C\# without switching to another language.
\section{Project Significance:}
The significance for this project are as followed
\begin{itemize}
	\item \textbf{Refreshing the Action Game Genre:}
	Many of the modern entries/players still adhere to this category of gaming and Proximity Assault on the other hand filled with strategic planning and humorous tone could attract wider range of audience.
	\item \textbf{Show Casing the Power of Unity:}
	This game serve as a proof about the informing developers about the power of unity by making this game more visually appealing and engaging in term of graphics.
	\item \textbf{Encourage Developers:}
	The development of this game will be documented and shared with public.
\end{itemize}

\section{Project Uniqueness:}
This game development project puts many aspects from real world in to this project.The game is an action based game.
Player navigating through different environment experiences different types of enemies.
\section{Project Scope:}
Project Scope is a map that guides the project keeping it on the track to the development journey, ensuring the team is on the track and in the end delivers an excellent product.
\subsection{Imagine:}
\begin{itemize}
\item 	You, the fearless hero navigating through the bustling areas of enemies.
\item Fighting with a diverse set of enemies in which each carries a  different set of abilities and power that can cause a lots of damage to you.
\item 	Equipping you with huge arsenal of bug-bashing tools, along with some special abilities from store and some new experimental gadgets from shop/inventory.
\end{itemize}
The project scope will clarify:
\begin{itemize}
	\item \textbf{What’s In:}We will have 10-15 hour of level design, 10-15 different enemies to fight with along with at least 8 upgradable  weapons/tools, Leaderboards, achievements, and an optional humorous touches add spice to the mix.
	\item \textbf{What's Out:}VR support, complex multiplayer features and extensive character customization won’t be part of the initial launch. The main focus will be on delivering the core gameplay first.
	\item \textbf{Why these Choices:} Budget, timeline and team size play a key role in these deliverables. We aim to allocate all the resources with great care and completely utilize them without compromising the project.
\end{itemize}
The project scope keeps the project in a fence. We can always adjust and adopt the boundaries, ensuring a fun and achievable development process. The table below summarize these features in a more understandable way.

\begin{tabular}{l|l|l}

\textbf{Elements}& \textbf{ What's Included} & \textbf{ What's Excluded} \\
\hline
Platforms & PC \& Mobile & VR \\ 
\hline
Engine & Unity & None \\
\hline
Game Play Hour & 10-15 & None \\
\hline
Environment & Diverse City Locations & None \\
\hline
Enemies & Increased with level & None \\
\hline
Extermination Tools & 8+ Up gradable Options & None \\
\hline

\end{tabular}


